\documentclass[man]{apa6}

\usepackage{amssymb,amsmath}
\usepackage{ifxetex,ifluatex}
\usepackage{fixltx2e} % provides \textsubscript
\ifnum 0\ifxetex 1\fi\ifluatex 1\fi=0 % if pdftex
  \usepackage[T1]{fontenc}
  \usepackage[utf8]{inputenc}
\else % if luatex or xelatex
  \ifxetex
    \usepackage{mathspec}
    \usepackage{xltxtra,xunicode}
  \else
    \usepackage{fontspec}
  \fi
  \defaultfontfeatures{Mapping=tex-text,Scale=MatchLowercase}
  \newcommand{\euro}{€}
\fi
% use upquote if available, for straight quotes in verbatim environments
\IfFileExists{upquote.sty}{\usepackage{upquote}}{}
% use microtype if available
\IfFileExists{microtype.sty}{\usepackage{microtype}}{}

% Table formatting
\usepackage{longtable, booktabs}
\usepackage{lscape}
% \usepackage[counterclockwise]{rotating}   % Landscape page setup for large tables
\usepackage{multirow}		% Table styling
\usepackage{tabularx}		% Control Column width
\usepackage[flushleft]{threeparttable}	% Allows for three part tables with a specified notes section
\usepackage{threeparttablex}            % Lets threeparttable work with longtable

% Create new environments so endfloat can handle them
% \newenvironment{ltable}
%   {\begin{landscape}\begin{center}\begin{threeparttable}}
%   {\end{threeparttable}\end{center}\end{landscape}}

\newenvironment{lltable}
  {\begin{landscape}\begin{center}\begin{ThreePartTable}}
  {\end{ThreePartTable}\end{center}\end{landscape}}

  \usepackage{ifthen} % Only add declarations when endfloat package is loaded
  \ifthenelse{\equal{\string man}{\string man}}{%
   \DeclareDelayedFloatFlavor{ThreePartTable}{table} % Make endfloat play with longtable
   % \DeclareDelayedFloatFlavor{ltable}{table} % Make endfloat play with lscape
   \DeclareDelayedFloatFlavor{lltable}{table} % Make endfloat play with lscape & longtable
  }{}%



% The following enables adjusting longtable caption width to table width
% Solution found at http://golatex.de/longtable-mit-caption-so-breit-wie-die-tabelle-t15767.html
\makeatletter
\newcommand\LastLTentrywidth{1em}
\newlength\longtablewidth
\setlength{\longtablewidth}{1in}
\newcommand\getlongtablewidth{%
 \begingroup
  \ifcsname LT@\roman{LT@tables}\endcsname
  \global\longtablewidth=0pt
  \renewcommand\LT@entry[2]{\global\advance\longtablewidth by ##2\relax\gdef\LastLTentrywidth{##2}}%
  \@nameuse{LT@\roman{LT@tables}}%
  \fi
\endgroup}


\ifxetex
  \usepackage[setpagesize=false, % page size defined by xetex
              unicode=false, % unicode breaks when used with xetex
              xetex]{hyperref}
\else
  \usepackage[unicode=true]{hyperref}
\fi
\hypersetup{breaklinks=true,
            pdfauthor={},
            pdftitle={},
            colorlinks=true,
            citecolor=blue,
            urlcolor=blue,
            linkcolor=black,
            pdfborder={0 0 0}}
\urlstyle{same}  % don't use monospace font for urls

\setlength{\parindent}{0pt}
%\setlength{\parskip}{0pt plus 0pt minus 0pt}

\setlength{\emergencystretch}{3em}  % prevent overfull lines


% Manuscript styling
\captionsetup{font=singlespacing,justification=justified}
\usepackage{csquotes}
\usepackage{upgreek}

 % Line numbering
  \usepackage{lineno}
  \linenumbers


\usepackage{tikz} % Variable definition to generate author note

% fix for \tightlist problem in pandoc 1.14
\providecommand{\tightlist}{%
  \setlength{\itemsep}{0pt}\setlength{\parskip}{0pt}}

% Essential manuscript parts

  \shorttitle{Reproducibility Report}


  \author{Gustavo Villca Ponce\textsuperscript{1}, MohammadHossein Haqiqatkhah\textsuperscript{2}, \& Sigert Ariens\textsuperscript{3}}

  % \def\affdep{{"", "", ""}}%
  % \def\affcity{{"", "", ""}}%

  \affiliation{
    \vspace{0.5cm}
          \textsuperscript{1} student number:r0292033\\
          \textsuperscript{2} student number:r0607671\\
          \textsuperscript{3} student number:  }



      \keywords{\\

      \indent  X
    }
  




\usepackage{amsthm}
\newtheorem{theorem}{Theorem}[section]
\newtheorem{lemma}{Lemma}[section]
\theoremstyle{definition}
\newtheorem{definition}{Definition}[section]
\newtheorem{corollary}{Corollary}[section]
\newtheorem{proposition}{Proposition}[section]
\theoremstyle{definition}
\newtheorem{example}{Example}[section]
\theoremstyle{definition}
\newtheorem{exercise}{Exercise}[section]
\theoremstyle{remark}
\newtheorem*{remark}{Remark}
\newtheorem*{solution}{Solution}
\begin{document}

\maketitle

\setcounter{secnumdepth}{0}



\subsection{Introducion}\label{introducion}

This is a statistical report, based on (add citation) for the class of
\emph{Statistics VI (Seminar on statistical analyses of psychological
research data) {[}P0Q01a{]}}. As required by the guidelines of this
project, this report will consist of three main parts, in which we will
try to 1)Check the reproducibility status of the published results ,
2)Check the robustness status of the confirmatory analyses and 3) Check
the pre-registration status of the published results by comparing the
pre-registered protocol to the published paper.

\subsection{Reproducibility}\label{reproducibility}

\subsubsection{Exploratory analysis}\label{exploratory-analysis}

To begin the replication portion of this report, we start by exploring
the possibility of a monotonic relationship between digital screen-time
and mental well-being as described by Przybylski and Weinstein (2017),
We achieve this my making use of the Besyan Information Criterion (BIC)
and comparing the simple linear models of all variables concerning
digital screen-time with their simple and quadratic counterparts. We
found that for the models that controlled for confounding variables, the
fit was greater for the models with both a linear and quadratic
component. For the unadjusted models, we found one model of which the
BIC index favoured a linear model alone. Specifically, for smartphone
use during weekdays ( see, Przybylski \& Weinstein (2017), figure 1).

\subsubsection{Confirmatory analysis}\label{confirmatory-analysis}

Following the steps described by the authors we start the exploratory
data analysis by creating quadratic models of all four types of digital
activities consisting of both linear and of non-linear components, next
we extracted all the important value( \emph{SD}, \emph{P}-values,
\emph{ß}, Confidence intervals and Cohen's \emph{d}) out of the models
and created two tables , the first table contains the outcome of the
models without taking into account the control variables described in
the paper, namely gender ,ethnicity and Socio Economical Status. The
second table contains the outcomes of the models with the control
variables (See tables below)

\begin{tabular}{l|r|r|r|r|r|r}
\hline
  & b & SE & CI(2.5\%) & CI(97.5\%) & p & d\\
\hline
Watch Weekday Linear & 0.99 & 0.10 & 0.79 & 1.20 & 0.00 & 0.06\\
\hline
Watch Weekday Quadratic & -0.14 & 0.01 & -0.16 & -0.12 & 0.00 & 0.09\\
\hline
Watch Weekend Linear & 1.55 & 0.10 & 1.36 & 1.74 & 0.00 & 0.10\\
\hline
Watch Weekend Quadratic & -0.17 & 0.01 & -0.18 & -0.15 & 0.00 & 0.13\\
\hline
Play Weekday Linear & 3.71 & 0.12 & 3.47 & 3.95 & 0.00 & 0.20\\
\hline
Play Weekday Quadratic & -0.34 & 0.01 & -0.37 & -0.32 & 0.00 & 0.18\\
\hline
Play Weekend Linear & 3.20 & 0.09 & 3.03 & 3.38 & 0.00 & 0.22\\
\hline
Play Weekend Quadratic & -0.27 & 0.01 & -0.28 & -0.25 & 0.00 & 0.20\\
\hline
Computer Weekday Linear & 1.32 & 0.10 & 1.11 & 1.52 & 0.00 & 0.08\\
\hline
Computer Weekday Quadratic & -0.17 & 0.01 & -0.19 & -0.15 & 0.00 & 0.11\\
\hline
Computer Weekend Linear & 1.60 & 0.09 & 1.42 & 1.78 & 0.00 & 0.11\\
\hline
Computer Weekend Quadratic & -0.18 & 0.01 & -0.19 & -0.16 & 0.00 & 0.14\\
\hline
Smatphone Weekday Linear & -0.56 & 0.09 & -0.73 & -0.40 & 0.00 & 0.04\\
\hline
Smatphone Weekday Quadratic & -0.01 & 0.01 & -0.03 & 0.00 & 0.11 & 0.01\\
\hline
Smatphone Weekend Linear & 0.46 & 0.08 & 0.29 & 0.62 & 0.00 & 0.03\\
\hline
Smatphone Weekend Quadratic & -0.10 & 0.01 & -0.11 & -0.08 & 0.00 & 0.08\\
\hline
\end{tabular}

\begin{tabular}{l|r|r|r|r|r|r}
\hline
  & b & SE & CI(2.5\%) & CI(97.5\%) & p & d\\
\hline
Watch Weekday Linear & 0.96 & 0.10 & 0.77 & 1.16 & 0.00 & 0.06\\
\hline
Watch Weekday Quadratic & -0.13 & 0.01 & -0.15 & -0.11 & 0.00 & 0.09\\
\hline
Watch Weekend Linear & 1.68 & 0.09 & 1.49 & 1.86 & 0.00 & 0.11\\
\hline
Watch Weekend Quadratic & -0.17 & 0.01 & -0.19 & -0.16 & 0.00 & 0.14\\
\hline
Play Weekday Linear & 0.27 & 0.12 & 0.02 & 0.51 & 0.03 & 0.01\\
\hline
Play Weekday Quadratic & -0.07 & 0.01 & -0.09 & -0.05 & 0.00 & 0.04\\
\hline
Play Weekend Linear & 0.55 & 0.09 & 0.37 & 0.74 & 0.00 & 0.04\\
\hline
Play Weekend Quadratic & -0.08 & 0.01 & -0.10 & -0.06 & 0.00 & 0.06\\
\hline
Computer Weekday Linear & 1.45 & 0.10 & 1.26 & 1.65 & 0.00 & 0.09\\
\hline
Computer Weekday Quadratic & -0.17 & 0.01 & -0.19 & -0.15 & 0.00 & 0.11\\
\hline
Computer Weekend Linear & 1.64 & 0.09 & 1.47 & 1.81 & 0.00 & 0.12\\
\hline
Computer Weekend Quadratic & -0.18 & 0.01 & -0.19 & -0.16 & 0.00 & 0.14\\
\hline
Smatphone Weekday Linear & 0.20 & 0.08 & 0.04 & 0.37 & 0.02 & 0.02\\
\hline
Smatphone Weekday Quadratic & -0.06 & 0.01 & -0.07 & -0.04 & 0.00 & 0.05\\
\hline
Smatphone Weekend Linear & 0.96 & 0.08 & 0.80 & 1.12 & 0.00 & 0.07\\
\hline
Smatphone Weekend Quadratic & -0.12 & 0.01 & -0.13 & -0.10 & 0.00 & 0.11\\
\hline
\end{tabular}

\begin{table}

\caption{(\#tab:table making)Table.1 Uncontrolled variables}
\centering
\begin{tabular}[t]{l|r|r|r|r|r|r}
\hline
  & b & SE & CI(2.5\%) & CI(97.5\%) & p & d\\
\hline
Watch Weekday Linear & 0.99 & 0.10 & 0.79 & 1.20 & 0.00 & 0.06\\
\hline
Watch Weekday Quadratic & -0.14 & 0.01 & -0.16 & -0.12 & 0.00 & 0.09\\
\hline
Watch Weekend Linear & 1.55 & 0.10 & 1.36 & 1.74 & 0.00 & 0.10\\
\hline
Watch Weekend Quadratic & -0.17 & 0.01 & -0.18 & -0.15 & 0.00 & 0.13\\
\hline
Play Weekday Linear & 3.71 & 0.12 & 3.47 & 3.95 & 0.00 & 0.20\\
\hline
Play Weekday Quadratic & -0.34 & 0.01 & -0.37 & -0.32 & 0.00 & 0.18\\
\hline
Play Weekend Linear & 3.20 & 0.09 & 3.03 & 3.38 & 0.00 & 0.22\\
\hline
Play Weekend Quadratic & -0.27 & 0.01 & -0.28 & -0.25 & 0.00 & 0.20\\
\hline
Computer Weekday Linear & 1.32 & 0.10 & 1.11 & 1.52 & 0.00 & 0.08\\
\hline
Computer Weekday Quadratic & -0.17 & 0.01 & -0.19 & -0.15 & 0.00 & 0.11\\
\hline
Computer Weekend Linear & 1.60 & 0.09 & 1.42 & 1.78 & 0.00 & 0.11\\
\hline
Computer Weekend Quadratic & -0.18 & 0.01 & -0.19 & -0.16 & 0.00 & 0.14\\
\hline
Smatphone Weekday Linear & -0.56 & 0.09 & -0.73 & -0.40 & 0.00 & 0.04\\
\hline
Smatphone Weekday Quadratic & -0.01 & 0.01 & -0.03 & 0.00 & 0.11 & 0.01\\
\hline
Smatphone Weekend Linear & 0.46 & 0.08 & 0.29 & 0.62 & 0.00 & 0.03\\
\hline
Smatphone Weekend Quadratic & -0.10 & 0.01 & -0.11 & -0.08 & 0.00 & 0.08\\
\hline
\end{tabular}
\end{table}

\subsubsection{Reproducibility analysis}\label{reproducibility-analysis}

Although we were able to extract all the important statistics from the
raw data without too many issues, notice that some of our values are
different from those reported in Przybylski and Weinstein (2017).
Specifically, we noticed two types of differences in both tables , the
first type are small one decimal differences, for example, in our
replication of their analysis we obtain a ß value of \emph{ß} =0.99 for
the linear component of the variable \enquote{watching films and tv
programs in weekdays}, whereas in their paper the authors reports a
value of \emph{ß}=.98, we see this occur not only for ß but for other
values too, such as the standard deviation of the linear component of
the variable \enquote{time spent playing games} (our \emph{SD}=0.12 vs
their \emph{SD}=.11 ), in the same model we obtain a
\emph{\textbar{}d\textbar{}} value of 0.20, compared to their
\emph{\textbar{}d\textbar{}} value of 19. These small one decimal
differences can be found in both tables, a potential explanation would
be a differences in the rounding of the number. This explanation becomes
less likely, however, once we take onto account the second type of
difference we encountered. We found decimal differences exceeding one
decimal, for example, in table 2 we observe a \_ ß\_=0.27 for the linear
component of \enquote{playing games in the weekdays} vs \emph{ß}=.21
reported in the paper. These multi-decimal differences can't be explain
fully by a rounding difference of the decimals. In order to find the
origin of the different outputs, we looked into the data used by the
authors for their SSPS analysis. We noticed a difference in the amount
of missing values(NA) between the raw data and the data used in their
SSPS analysis, which will be elaborated on in the preregistration
section. It is likely that the researcher handled the missing values in
a way that wasn't reported in the paper, making it hard for us to fully
replicate the results without any differences. Furthermore, the way the
variables were coded in an ambiguous manner, making it difficult to
determine which specific variables were used in the models reported by
the researchers. Overall, there was a lack of clarity in crucial data
processing steps such as missing values and variable identification.

\subsubsection{Preregistration}\label{preregistration}

The data were acquired according to the specifications made by the
authors in the preregistration document. However, in the technical
report on their OSF page, the authors said to use a 3\% margin of error
at the 95\%CI to estimate sample size. In the published paper, the
authors report a 0.3\% margin of error, arriving at the same estimate of
sample size (N = 298,080. Furthermore, The authors reported a total n of
120,115 participants with usable data. When we attempted to replicate
their analyses, we met with a further reduction of n to 98278. This is
not reported anywhere in the published article. Finally, the two data
documents provided by the authors differ in the amount of NA data they
contain. Where the .csv file contains 21837 rows with missing values,
the .sav file contains 44642 rows with missing values. None of these
inconsistencies were reported by the authors in the final paper, and it
is unclear which data set the authors ultimately used in their analysis.
The fact that the data are ambiguous is a major obstacle for replication
analyses.

The preregistration document stated that testing the displacement
hypothesis was to be done by linear regression modelling, predicting
mental well-being through composite scores of screen time. The authors
did not conduct these analyses. They explain that \enquote{Interocular}
tests were sufficient to exclude this hypothesis. Although we only found
one unadjusted model for which the BIC criterum favours a purely linear
model, this does allow one to question why the authors refrained from
the formal hypothesis test detailed in the preregistration document.

The authors ignored the measure of summed screen time which they
included in their preregistration document. The authors reported this
accordingly, although our above analysis allows the questionability of
their deviation on this point. Finally, the authors did not concretely
specify what particular coding they intended to use for the control
variables of \enquote{whether living in a deprived area}, and
\enquote{whether black and minority ethnicity}. While the conditionality
of the term \enquote{whether} implies binary coding, the subsequent
reference to the specific questions would also allow one to assume that
the authors used the values from those questions. This is relevant
because other codings of these variables are also present in the data.
Although this is a minor issue, a more clear issue is also present. The
authors noted in their deviations from analysis plan that the
preregistered control variables of \enquote{whether parents married} and
\enquote{whether native born}, they did not include the omission of
these variables in the published article.

\newpage

\section{References}\label{references}

\begingroup
\setlength{\parindent}{-0.5in} \setlength{\leftskip}{0.5in}

\hypertarget{refs}{}

\endgroup






\end{document}
