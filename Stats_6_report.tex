\documentclass[man]{apa6}

\usepackage{amssymb,amsmath}
\usepackage{ifxetex,ifluatex}
\usepackage{fixltx2e} % provides \textsubscript
\ifnum 0\ifxetex 1\fi\ifluatex 1\fi=0 % if pdftex
  \usepackage[T1]{fontenc}
  \usepackage[utf8]{inputenc}
\else % if luatex or xelatex
  \ifxetex
    \usepackage{mathspec}
    \usepackage{xltxtra,xunicode}
  \else
    \usepackage{fontspec}
  \fi
  \defaultfontfeatures{Mapping=tex-text,Scale=MatchLowercase}
  \newcommand{\euro}{€}
\fi
% use upquote if available, for straight quotes in verbatim environments
\IfFileExists{upquote.sty}{\usepackage{upquote}}{}
% use microtype if available
\IfFileExists{microtype.sty}{\usepackage{microtype}}{}

% Table formatting
\usepackage{longtable, booktabs}
\usepackage{lscape}
% \usepackage[counterclockwise]{rotating}   % Landscape page setup for large tables
\usepackage{multirow}		% Table styling
\usepackage{tabularx}		% Control Column width
\usepackage[flushleft]{threeparttable}	% Allows for three part tables with a specified notes section
\usepackage{threeparttablex}            % Lets threeparttable work with longtable

% Create new environments so endfloat can handle them
% \newenvironment{ltable}
%   {\begin{landscape}\begin{center}\begin{threeparttable}}
%   {\end{threeparttable}\end{center}\end{landscape}}

\newenvironment{lltable}
  {\begin{landscape}\begin{center}\begin{ThreePartTable}}
  {\end{ThreePartTable}\end{center}\end{landscape}}

  \usepackage{ifthen} % Only add declarations when endfloat package is loaded
  \ifthenelse{\equal{\string man}{\string man}}{%
   \DeclareDelayedFloatFlavor{ThreePartTable}{table} % Make endfloat play with longtable
   % \DeclareDelayedFloatFlavor{ltable}{table} % Make endfloat play with lscape
   \DeclareDelayedFloatFlavor{lltable}{table} % Make endfloat play with lscape & longtable
  }{}%



% The following enables adjusting longtable caption width to table width
% Solution found at http://golatex.de/longtable-mit-caption-so-breit-wie-die-tabelle-t15767.html
\makeatletter
\newcommand\LastLTentrywidth{1em}
\newlength\longtablewidth
\setlength{\longtablewidth}{1in}
\newcommand\getlongtablewidth{%
 \begingroup
  \ifcsname LT@\roman{LT@tables}\endcsname
  \global\longtablewidth=0pt
  \renewcommand\LT@entry[2]{\global\advance\longtablewidth by ##2\relax\gdef\LastLTentrywidth{##2}}%
  \@nameuse{LT@\roman{LT@tables}}%
  \fi
\endgroup}


\ifxetex
  \usepackage[setpagesize=false, % page size defined by xetex
              unicode=false, % unicode breaks when used with xetex
              xetex]{hyperref}
\else
  \usepackage[unicode=true]{hyperref}
\fi
\hypersetup{breaklinks=true,
            pdfauthor={},
            pdftitle={Assigment Statistics 6},
            colorlinks=true,
            citecolor=blue,
            urlcolor=blue,
            linkcolor=black,
            pdfborder={0 0 0}}
\urlstyle{same}  % don't use monospace font for urls

\setlength{\parindent}{0pt}
%\setlength{\parskip}{0pt plus 0pt minus 0pt}

\setlength{\emergencystretch}{3em}  % prevent overfull lines


% Manuscript styling
\captionsetup{font=singlespacing,justification=justified}
\usepackage{csquotes}
\usepackage{upgreek}

 % Line numbering
  \usepackage{lineno}
  \linenumbers


\usepackage{tikz} % Variable definition to generate author note

% fix for \tightlist problem in pandoc 1.14
\providecommand{\tightlist}{%
  \setlength{\itemsep}{0pt}\setlength{\parskip}{0pt}}

% Essential manuscript parts
  \title{Assigment Statistics 6}

  \shorttitle{Title}


  \author{Gustavo Villca Ponce\textsuperscript{1}~\& Ernst-August Doelle\textsuperscript{1,2}}

  % \def\affdep{{"", ""}}%
  % \def\affcity{{"", ""}}%

  \affiliation{
    \vspace{0.5cm}
          \textsuperscript{1} Wilhelm-Wundt-University\\
          \textsuperscript{2} Konstanz Business School  }

  \authornote{
    Add complete departmental affiliations for each author here. Each new
    line herein must be indented, like this line.
    
    Enter author note here.
    
    Correspondence concerning this article should be addressed to Gustavo
    Villca Ponce, Postal address. E-mail:
    \href{mailto:my@email.com}{\nolinkurl{my@email.com}}
  }


  \abstract{Enter abstract here. Each new line herein must be indented, like this
line.}
  \keywords{Blood and tears \\

    \indent Word count: X
  }





\usepackage{amsthm}
\newtheorem{theorem}{Theorem}[section]
\newtheorem{lemma}{Lemma}[section]
\theoremstyle{definition}
\newtheorem{definition}{Definition}[section]
\newtheorem{corollary}{Corollary}[section]
\newtheorem{proposition}{Proposition}[section]
\theoremstyle{definition}
\newtheorem{example}{Example}[section]
\theoremstyle{definition}
\newtheorem{exercise}{Exercise}[section]
\theoremstyle{remark}
\newtheorem*{remark}{Remark}
\newtheorem*{solution}{Solution}
\begin{document}

\maketitle

\setcounter{secnumdepth}{0}



\#\#introducion

This is a statistical report, based on (add citation) for the class of
\emph{Statistics VI (Seminar on statistical analyses of psychological
research data) {[}P0Q01a{]}}. As required by the guidelines of this
project, this report will consist of three main parts, in which we will
try to 1)Check the reproducibility status of the published results ,
2)Check the robustness status of the confirmatory analyses and 3) Check
the pre-registration status of the published results by comparing the
pre-registered protocol to the published paper.

\hypertarget{reproducibility}{%
\subsection{Reproducibility}\label{reproducibility}}

\hypertarget{exploratory-analysis}{%
\subsubsection{Exploratory analysis}\label{exploratory-analysis}}

To begin the replication portion of this report, we start by exploring
the possibility of a monotonic relationship between digital screen-time
and mental well-being as described by Przybylski and Weinstein (2017),We
achieve this my making use of the Besyan Information Criterion (BIC) and
comparing the simple linear models of all variables concerning digital
screen-time with their quadratic counterparts. From these results ( add
here ) we can suggest that a simple linear regression model would fit
beter at least the variables of \enquote{time using computer} and
\enquote{time using smartphone}. However, this is consistent with what
is reported by Przybylski and Weinstein (2017), the authors claim that
although a monotonic relationship could possibly be fitted onto the
data, it would be very unsuitable, this can be confirmed by observing
the plotted data presented by the researchers ( see, Przybylski \&
Weinstein (2017), figure 1)

\hypertarget{confirmatory-analysis}{%
\subsubsection{Confirmatory analysis}\label{confirmatory-analysis}}

\$\$\$\$

Following the steps described by the authors we start the exploratory
data analysis by creating quadratic models of all four types of digital
activities consisting of both linear and of non-linear components, next
we extracted all the important value( \emph{SD}, \emph{P}-values,
\emph{ß1}, Confidence intervals and Cohen's \emph{d}) out of the models
and created two tables , the first table contains the outcome of the
models without taking into account the control variables described in
the paper, namely gender ,ethnicity and Socio Economical Status. The
second table contains the outcomes of the models with the control
variables (See tables below)

\begin{tabular}{l|r|r|r|r|r|r}
\hline
  & b & SE & CI(2.5\%) & CI(97.5\%) & p & d\\
\hline
Watch Weekday Linear & 0.99 & 0.10 & 0.79 & 1.20 & 0.00 & 0.06\\
\hline
Watch Weekday Quadratic & -0.14 & 0.01 & -0.16 & -0.12 & 0.00 & 0.09\\
\hline
Watch Weekend Linear & 1.55 & 0.10 & 1.36 & 1.74 & 0.00 & 0.10\\
\hline
Watch Weekend Quadratic & -0.17 & 0.01 & -0.18 & -0.15 & 0.00 & 0.13\\
\hline
Play Weekday Linear & 3.71 & 0.12 & 3.47 & 3.95 & 0.00 & 0.20\\
\hline
Play Weekday Quadratic & -0.34 & 0.01 & -0.37 & -0.32 & 0.00 & 0.18\\
\hline
Play Weekend Linear & 3.20 & 0.09 & 3.03 & 3.38 & 0.00 & 0.22\\
\hline
Play Weekend Quadratic & -0.27 & 0.01 & -0.28 & -0.25 & 0.00 & 0.20\\
\hline
Computer Weekday Linear & 1.32 & 0.10 & 1.11 & 1.52 & 0.00 & 0.08\\
\hline
Computer Weekday Quadratic & -0.17 & 0.01 & -0.19 & -0.15 & 0.00 & 0.11\\
\hline
Computer Weekend Linear & 1.60 & 0.09 & 1.42 & 1.78 & 0.00 & 0.11\\
\hline
Computer Weekend Quadratic & -0.18 & 0.01 & -0.19 & -0.16 & 0.00 & 0.14\\
\hline
Smatphone Weekday Linear & -0.56 & 0.09 & -0.73 & -0.40 & 0.00 & 0.04\\
\hline
Smatphone Weekday Quadratic & -0.01 & 0.01 & -0.03 & 0.00 & 0.11 & 0.01\\
\hline
Smatphone Weekend Linear & 0.46 & 0.08 & 0.29 & 0.62 & 0.00 & 0.03\\
\hline
Smatphone Weekend Quadratic & -0.10 & 0.01 & -0.11 & -0.08 & 0.00 & 0.08\\
\hline
\end{tabular}

\begin{tabular}{l|r|r|r|r|r|r}
\hline
  & b & SE & CI(2.5\%) & CI(97.5\%) & p & d\\
\hline
Watch Weekday Linear & 0.99 & 0.10 & 0.79 & 1.20 & 0.00 & 0.06\\
\hline
Watch Weekday Quadratic & -0.14 & 0.01 & -0.16 & -0.12 & 0.00 & 0.09\\
\hline
Watch Weekend Linear & 1.55 & 0.10 & 1.36 & 1.74 & 0.00 & 0.10\\
\hline
Watch Weekend Quadratic & -0.17 & 0.01 & -0.18 & -0.15 & 0.00 & 0.13\\
\hline
Play Weekday Linear & 3.71 & 0.12 & 3.47 & 3.95 & 0.00 & 0.20\\
\hline
Play Weekday Quadratic & -0.34 & 0.01 & -0.37 & -0.32 & 0.00 & 0.18\\
\hline
Play Weekend Linear & 3.20 & 0.09 & 3.03 & 3.38 & 0.00 & 0.22\\
\hline
Play Weekend Quadratic & -0.27 & 0.01 & -0.28 & -0.25 & 0.00 & 0.20\\
\hline
Computer Weekday Linear & 1.32 & 0.10 & 1.11 & 1.52 & 0.00 & 0.08\\
\hline
Computer Weekday Quadratic & -0.17 & 0.01 & -0.19 & -0.15 & 0.00 & 0.11\\
\hline
Computer Weekend Linear & 1.60 & 0.09 & 1.42 & 1.78 & 0.00 & 0.11\\
\hline
Computer Weekend Quadratic & -0.18 & 0.01 & -0.19 & -0.16 & 0.00 & 0.14\\
\hline
Smatphone Weekday Linear & -0.56 & 0.09 & -0.73 & -0.40 & 0.00 & 0.04\\
\hline
Smatphone Weekday Quadratic & -0.01 & 0.01 & -0.03 & 0.00 & 0.11 & 0.01\\
\hline
Smatphone Weekend Linear & 0.46 & 0.08 & 0.29 & 0.62 & 0.00 & 0.03\\
\hline
Smatphone Weekend Quadratic & -0.10 & 0.01 & -0.11 & -0.08 & 0.00 & 0.08\\
\hline
\end{tabular}

\#\#\#Reproducibility analysis Although we were able to extract all the
important statistics out of the raw data without to many issues, notice
that some of our values are different from those reported in Przybylski
and Weinstein (2017). More specifically, we can see differences in

\newpage

\hypertarget{references}{%
\section{References}\label{references}}

\begingroup
\setlength{\parindent}{-0.5in}
\setlength{\leftskip}{0.5in}

\hypertarget{refs}{}

\endgroup






\end{document}
